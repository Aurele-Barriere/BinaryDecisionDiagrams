\documentclass[12pt]{article}
\usepackage[utf8]{inputenc}
\usepackage[T1]{fontenc}
\usepackage[french]{babel}
\usepackage{amsmath,amsfonts,amssymb}
%\usepackage{fullpage}
\usepackage{graphicx}
\usepackage{hyperref}
\usepackage{listings}

\title{Binaey Decision Diagrams}
\author{Aurèle Barrière \& Jérémy Thibault}
\date{6 mai 2016}


\def\question#1{\subsection*{Question #1}}
\def\phix{\varphi\uparrow\up{x}}
\begin{document}
\maketitle
%\newpage

\section*{2\ BDD et forme normale if-then-else}

\question{1}

On a $\varphi\equiv\phix$ ssi pour toute valuation $V$, $V(\varphi) = V(\phix)$.

Soit $V$ une valuation quelconque. 

\paragraph{Supposons dans un premier temps $V(x) = 1$.} Comme $\phix = (x \wedge\varphi[1/x])\vee(\neg x\wedge\varphi[0/x])$ par définition, on a $V(\phix) = V(\varphi[1/x])$.

De même, comme $V(x) = 1$, on peut remplacer $x$ par 1 dans $\varphi$ sans changer $V(\varphi)$. En effet, par la définition inductive de la valuation d'une formule est un prolongement de la valuation sur les variables. Ainsi $V(\varphi)=V(\varphi[1/x])$ quand $V(x)=1)$.

Ainsi, dans ce cas, on a bien $V(\varphi) = V(\phix)$.

\paragraph{Désormais, supposons $V(x)=0$.} Cette fois-ci, $V(\phix) = V(\varphi[0/x])$.

De même, $V(\varphi) =  V(\varphi[0/x]) = V(\phix)$.

Ainsi, on a bien $\varphi\equiv\phix$.

\question{2}
\question{3}


\end{document}
