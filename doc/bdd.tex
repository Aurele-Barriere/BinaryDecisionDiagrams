\documentclass[12pt]{article}
\usepackage[utf8]{inputenc}
\usepackage[T1]{fontenc}
\usepackage[french]{babel}
\usepackage{amsmath,amsfonts,amssymb}
%\usepackage{fullpage}
\usepackage{graphicx}
\usepackage{hyperref}
\usepackage{listings}

\title{Binary Decision Diagrams}
\author{Aurèle Barrière \& Jérémy Thibault}
\date{6 mai 2016}


\def\question#1{\subsection*{Question #1}}
\def\phix{\varphi\uparrow\up{x}}
\def\ite{\textit{if-then-else}}
\def\P{\mathcal{P}}
\def\code#1{\texttt{#1}}
\begin{document}
\maketitle
%\newpage

\section*{2\ BDD et forme normale \ite}

\question{1}

On a $\varphi\equiv\phix$ ssi pour toute valuation $V$, $V(\varphi) = V(\phix)$.

Soit $V$ une valuation quelconque.

\paragraph{Supposons dans un premier temps $V(x) = 1$.} Comme $\phix = (x \wedge\varphi[1/x])\vee(\neg x\wedge\varphi[0/x])$ par définition, on a $V(\phix) = V(\varphi[1/x])$.

De même, comme $V(x) = 1$, on peut remplacer $x$ par 1 dans $\varphi$ sans changer $V(\varphi)$. En effet, par la définition inductive de la valuation d'une formule est un prolongement de la valuation sur les variables. Ainsi $V(\varphi)=V(\varphi[1/x])$ quand $V(x)=1)$.

Ainsi, dans ce cas, on a bien $V(\varphi) = V(\phix)$.

\paragraph{Désormais, supposons $V(x)=0$.} Cette fois-ci, $V(\phix) = V(\varphi[0/x])$.

De même, $V(\varphi) =  V(\varphi[0/x]) = V(\phix)$.

Ainsi, on a bien $\varphi\equiv\phix$.

\question{2}

On commence par montrer que toute formule peut être mise sous forme d'un arbre de décision.

En effet, soit $\varphi$ une formule faisant intervenir $n$ variables. On choisit un ordre sur ces variables que l'on nomme $x_1\dots x_n$.

On construit l'arbre de décision suivant : la racine est $x_1$, et le fils faible et fort de $x_i$ sont $x_{(i+1)}$ pour $i\in\{1\dots n-1\}$. Enfin les fils faibles et forts des n\oe uds étiquetés $x_n$ sont la valeur de $\varphi$ par la valuation qui à $x_i$ associe 1 si pour aboutir à cette feuille il faut prendre le fils fort à la $i$\up{ème} ligne, et 0 sinon.
%j'espere que c'est clair, c'est un truc tout ballot mais bon faut l'expliquer

On remarque également que tout sous-arbre d'un arbre de décision reste un arbre de décision. %utile pour l'hérédité

On va ensuite construire la forme \ite\ à partir de cet arbre de décision.

On va montrer que toute formule est équivalente à une formule \ite\ par récurrence sur le nombre de variables impliquées dans la formule.

On pose $\P(n)$ la propriété : ``Toute formule à $n$ variables est équivalente à une formule \ite''.

\paragraph{Initialisation : }
Pour $n=0$, la formule est nécessairement constante quelle que soit la valuation, égale à 0 ou 1. On peut donc prendre la formule \ite\ 0 ou 1 correspondante.

Pour $n=1$, l'arbre de décision de la formule construit comme précédemment est ainsi réduit à 1 n\oe ud interne (étiqueté par la variable utilisée, appelée $x$), dont les feuilles sont l'évaluation de $\varphi [ 1/x ]$ et $\varphi [ 0/x ]$ : 0 ou 1.

Dans ce cas, la formule est équivalent à $\phix$. C'est bien équivalent par la question précédente, et c'est bien une formule \ite\ car elle est formée à partir de l'opérateur \ite, des constantes 0 et 1 et que la formule test est $x$, une variable non niée.

\paragraph{Hérédité : }
On suppose $\P (n)$ vraie. Montrons $\P (n+1)$.

Soit $\varphi$ une formule utilisant $n+1$ variables $x_1\dots x_{n+1}$. Son arbre de décision construit comme précédemment a pour racine $x_1$, dont les fils faibles et forts sont les racines de sous-arbres de décision de hauteur $n$ (et de racine étiquetée par $x_2$). Ces deux sous-arbres représentent donc deux formule équivalentes à deux formes \ite\ par hypothèse de récurrence. Ces formules \ite\ seront notées $f$ et $F$,

On peut donc construire la formule \ite\ suivante : $x_1\rightarrow f,F$. C'est bien une formule \ite\ : $f$ et $F$ le sont, et on a utilisé l'opérateur \ite\ avec en formule test $x_1$, une variable non niée.

De plus, $\varphi$ est bien équivalent à cette formule, puisque dans les feuilles de $f$ on considère les valuations où $x_1 = 0$ et dans $F$ les valuations où $x_1 = 1$.


\paragraph{En conclusion} on a montré par récurrence $\P (n)\forall n\in\mathbb{N}$.

De plus, comme toute formule implique un nombre finie de variables (étant elle-même finie), alors toute formule est équivalente à une formule \ite.

\question{3}
Montrons l'existence et l'unicité, pour toute formule, d'un ROBDD équivalent.

Nous considérons une formule $f$.
\paragraph{Existence :}
Proposons un algorithme pour construire un ROBDD à partir d'une formule. Si cet algorithme est correct, alors on aura bien l'existence d'un ROBDD pour représenter toute formule.

Soit $f$ une formule, utilisant les variables $x_1\dots x_n$. Comme vu précédemment, on peut la mettre sous forme d'un arbre de décision ordonné, dont la racine est $x_1$, de fils $x_2$ \dots.

On va alors transformer cet arbre de décision en un ROBDD. Il faut conserver l'ordre des variables et supprimer tout couple de n\oe uds ayant même étiquette et mêmes fils, puis vérifier la propriété d'utilité.

On commence par fusionner toutes les feuilles 0, puis toutes les feuilles 1.

Ensuite, on regarde les n\oe uds étiquetés par $x_n$. Lorsque deux n\oe uds ont les mêmes fils, on les fusionne : on n'a plus qu'un seul n\oe uds, dont les fils sont ceux des deux n\oe uds initiaux, et vers lequel on fait pointer toutes les flèches qui pointaient vers les n\oe uds initiaux.

Ensuite, on regarde ceux étiquetés par $x_{n-1}$ et on fait la même chose.

On continue jusqu'à la racine.

On dispose donc d'un arbre de décision ordonné, tel qu'il n'existe plus deux n\oe uds ayant la même étiquette et les même fils. On a la propriété d'unicité. Cependant, il reste la condition d'utilité à vérifier.

On parcourt donc le graphe obtenu, et dès qu'on voit un n\oe ud dont les fils faibles et forts sont égaux, on le supprime et on fait pointer tout ce qui pointait vers lui vers son fils. Cela ne change pas le reste du graphe donc on peut le faire dans l'ordre que l'on veut, tant qu'il reste de tels n\oe uds. Cela ne change pas non plus l'ordre des variables.

Cependant, on a peut être compromis la condition d'unicité. On reparcourt alors en partant du bas pour supprimer les n\oe uds uniques. Puis on enlève encore les n\oe uds inutiles. Et on itère ces deux opérations jusqu'à stabilisation.

L'algorithme termine puisque chacune des opérations fait diminuer strictement le nombre de n\oe uds et que le graphe initial est fini.
L'algorithme est correct puisqu'à la fin, on satisfait les deux conditions d'unicité et d'utilité : on a donc bien un ROBDD. Enfin, on n'a à aucun moment altéré la formule représentée par le graphe : toutes ces opérations sont licites sémantiquement. %bizarrement formulé?

Ainsi, pour toute formule, avec un ordre donné, il existe un ROBDD qui représente cette formule.

\paragraph{Unicité :}
Nous allons montrer l'unicité par récurrence sur $n$, le nombre de variables impliquées dans une formule.

\subparagraph{Initialisation :}
$n = 0$ Lorsque la formule $f$ n'utilise aucune variable, alors elle est équivalente à 0 ou 1. L'unique ROBBD correspondant consiste donc en un seul n\oe ud.

\subparagraph{Hérédité :}
Supposons l'unicité pour toute formule utilisant $n$ variables. Considérons maintenant la formule $f$ en utilisant $n+1$, notées $x_1\dots x_{n+1}$.

On peut se trouver dans deux situations, selon si la variable $x_1$ est utile ou non dans la formule.

Si $x_1$ n'est pas utile, c'est-à-dire que $f[1/x_1]\equiv f[0/x_1]$, alors $x_1$ n'apparaît pas dans un ROBDD représentant $f$. Alors $f$ est une fonction de $n$ variables : $x_2\dots x_{n+1}$. Alors par hypothèse de récurrence, il existe un unique ROBDD représentant $f$.

Si la valuation de $x_1$ importe dans la valuation de $f$, alors il faut que $x_1$ apparaisse dans tout ROBDD représentant $f$. De plus, il faut qu'il apparaisse à la racine dans un ROBDD, sinon on rencontrerait une autre variable avant $x_1$ et on ne serait plus ordonné. On est de plus garantis que $x_1$ n'apparaît pas en dessous, puisqu'en dessous de la racine on a fixé la valeur de $x_1$.

Considérons alors deux ROBDDs représentant $f$ : $R_1$ et $R_2$. Ces deux ROBDD ont donc comme racine un n\oe ud étiqueté par $x_1$. Considérons désormais les fils de $x_1$ dans les deux arbres. Ces fils sont des ROBDD à $n$ variables, avec peut-être des n\oe uds communs.

Notons $l_1 = low(x_1)$ dans $R_1$,  $h_1 = high(x_1)$ dans $R_1$, $l_2 = low(x_1)$ dans $R_2$,  $h_2 = high(x_1)$ dans $R_2$.

Alors les graphes qui ont pour racines $l_1, l_2, h_1, h_2$ sont tous des ROBDD de formules à $n$ variables. Alors par hypothèse de récurrence ils sont uniques. Ainsi, comme $l_1$ et $l_2$ décrivent tous deux la formule $f[0/x_1]$, on a $l_1=l_2$. De même, $h_1=h_2$.

Cependant, dans $R_1$ et $R_2$, ces sous-ROBDD sont imbriqués : $l_1$ et $h_1$ partagent peut-être des n\oe uds.

Il faut donc montrer qu'il n'y a qu'une seule manière d'imbriquer deux ROBDD utilisant les mêmes variables dans le même ordre.
%c'est ca le point ``tricky''

Or c'est bien le cas : pour chaque variable $x_i$, $i\in[2\dots n]$ (ou pour chaque n\oe ud 0 et 1), lorsque deux n\oe uds étiquetés par $x_i$ ont les même fils dans un ROBDD comme dans l'autre, on les fusionne. Et on recommence tant qu'il existe de tels n\oe uds. Ce procédé ne peut conduire qu'à un seul ROBDD (quel que soit l'ordre dans lequel on considère les variables à fusionner), puisqu'une fusion ne modifie pas les graphes en dehors du n\oe ud fusionné, et ce procédé termine puisqu'il y a un nombre fini de n\oe uds. Enfin, ce procédé donne bien un ROBDD puisqu'il élimine tous les n\oe uds semblables.

Il n'y a donc bien qu'une seule manière d'imbriquer $l_1$ et $h_1$. Donc $R_1$ et $R_2$ sont identiques. Donc il existe un unique ROBDD pour décrire une formule à $n+1$ variables.

\subparagraph{Conclusion :} Ainsi, quel que soit le nombre de variables utilisées par une formule $f$, pour un ordre donné de ces variables il existe un unique ROBDD qui représente $f$.


\section*{4. Implémentation de ROBDDs}
\subsection*{Make}
Pour la fonction \code{make}, il y a deux possibilités. Si le n\oe ud existe déjà, il suffit de renvoyer son identifiant.
Si ce n'est pas le cas, il faut l'ajouter dans les tables, à condition que \code{low} et \code{high} soient différent. En effet, s'ils sont identiques, il est possible de simplifier le n\oe ud et renvoyer \code{low}.
% j'arrive pas a mieux xpliquer

\subsection*{Apply\_neg}
Pour la fonction \code{apply\_neg}, nous avons simplement implémenté la propriété qui nous était donnée. Il s'agit donc d'un parcours récursif de la formule, qui s'arrête une fois arrivé aux noeuds 0 et 1, qu'il faut inverser.

\subsection*{Apply}
Pour la fonction \code{apply}, il y a plusieurs cas.
Si les deux bdds sont des n\oe uds 1 ou 0, alors il suffit d'appliquer directement l'opérateur.

Sinon, si les deux étiquettes des racines sont différentes, il faut prendre l'étiquette la plus petite (qui vient en premier dans l'ordre des variables, afin de préserver la propriété d'ordonnancement) et construire le bdd qui a pour racine cette variable et comme fils les arbres construits de manière récursive grâce à la propriété.

Enfin, si les deux étiquettes sont égales, alors on peut simplifier et n'en garder qu'une dans le graphe final.

On aurait pu utiliser la programmation dynamique. Nous aurions alors dû stocker dans une table d'association par exemple, les liens entre formules déjà appliquées et leur identifiant.

\subsection*{Build}
La fonction \code{build} consiste en essence à des appels à \code{apply} avec le bon opérateur. La fonction utilise la définition inductive d'une formule.
Nous avons fait le choix, pour simplifier, de transformer toute implication en une négation et une disjonction. En effet, les deux étant équivalentes, le ROBDD construit sera le même.

\subsection*{Sat}
Une formule est satisfiable s'il existe au moins une valuation qui la rend vraie. Dans le ROBDD, une valuation correspond à un chemin dans le graphe. Si cette valuation existe, alors il existe un chemin qui mène au n\oe ud 1 dans le graphe.

Alors il suffit de parcourir le graphe récursivement à la recherche d'un 1.

\subsection*{Valid}
Pour savoir si une fonction est une tautologie, il ne faut pas qu'une valuation puisse rendre fausse la formule. Il ne faut donc pas qu'il y ait de 0 dans le ROBDD. La fonction \code{valid} parcourt donc de même le graphe à la recherche d'un 0.

Notons également qu'une formule valide sera naturellement strictement équivalente à 1, et donc aura le même ROBDD que 1 par unicité du ROBDD. Cela nous aurait permis de vérifier la validité bien plus simplement, et avec une complexité en temps constante. De même, une formule est satisfiable si et seulement si son ROBDD n'est pas 0.

\subsection*{Anysat}
Pour renvoyer une valuation qui satisfait la formule, nous parcourons en profondeur le graphe. Au fur et à mesure, une liste de nos choix de chemins est créée: c'est une valuation. Si le noeud 0 est atteint, un mauvais choix a été effectué, et il faut donc essayer une autre valuation. Si le noeud 1 est atteint, alors une valuation qui satisfait la formule est trouvée.

\section*{4.6 Application : les $n$ dames}

\subsection*{Construction de la formule}
La résolution doit se découper en plusieurs étapes : création de la formule, traduction en ROBDD, résolution et affichage.

La traduction en ROBDD se fait directement avec la fonction \code {build}, la résolution avec la fonction \code{anysat}. Enfin, nous avons créé une fonction qui affiche l'échiquier.

Pour traduire le problème en formule, nous avons choisi de prendre une variable par case de l'échiquier. Cette variable vaudra 1 si et seulement si il y a une dame sur cette case.

Il faut alors traduire les contraintes. Pour chaque ligne, il faut qu'il y ait une et une seule dame. Cela correspond à la formule suivante :
pour la ligne $i$,
$$\bigvee_{k=0}^{n-1} \left( x_{i,k}\wedge\bigwedge_{l=0,\ l\neq k}^{n-1} \neg x_{i,l}\right)$$
De même pour chaque colonne $j$,
$$\bigvee_{k=0}^{n-1} \left( x_{k,j}\wedge\bigwedge_{l=0,\ l\neq k}^{n-1} \neg x_{l,j}\right)$$

Pour les diagonales, il n'y a pas nécessairement une dame par diagonale. Cependant, pour la k\up{ième} diagonale, il faut préciser pour chaque case de cette diagonale que s'il y a une dame ici, alors il n'y en a pas sur toutes les autres cases de la diagonale.

Il nous faut alors un moyen de paramétrer les diagonales pour parcourir les cases y appartenant.

Pour les diagonales descendantes, pour la k\up{ième} diagonale (ou $k\in[0\dots 2\times n-1]$, $x_{i,j}$ appartient à la diagonale ssi $i+j=k$. De plus, la diagonale commence en abscisse à l'indice $max(0,k-(n-1))$ et termine à l'indice $min(k,n-1)$. %ajouter un dessin? ou est ce que c'est suffisament clair?

Nous avons donc les formules suivantes à ajouter :
$$\bigwedge_{i=max(0,k-(n-1))}^{min(k,n-1)} \left( \neg x_{i, k-i} \vee\bigwedge_{l=max(0,k-(n-1)),\ l\neq i}^{min(k,n-1)} \neg x_{l, k-l}\right)$$


Ensuite, pour les diagonales montantes, pour la k\up{ième} diagonale (ou $k\in[0\dots 2\times n-1]$, $x_{i,j}$ appartient à la diagonale ssi $j=i-k+n-1$.
Nous avons donc les formules suivantes à ajouter :
$$\bigwedge_{i=max(0,k-(n-1))}^{min(k,n-1)} \left( \neg x_{i, i-k+n-1} \vee\bigwedge_{l=max(0,k-(n-1)),\ l\neq i}^{min(k,n-1)} \neg x_{l, l-k+n-1}\right)$$

On peut se passer des formules pour les premières et dernières diagonales. En effet, elles ne comportent qu'une seule case.

Ainsi, la construction du ROBDD se fait à partir de la formule obtenue. Il suffit alors d'exécuter \code{anysat} pour obtenir une valuation qui satisfait le problème.

\subsection*{Résultats}
\begin{lstlisting}
1 queens
1
2 queens
not satisfiable
3 queens
not satisfiable
4 queens
0100
0001
1000
0010
5 queens
00100
00001
01000
00010
10000
6 queens
001000
000001
010000
000010
100000
000100
7 queens
0001000
0000001
0010000
0000010
0100000
0000100
1000000
8 queens
00100000
00000100
00010000
01000000
00000001
00001000
00000010
10000000
\end{lstlisting}

Pour 8 dames, l'exécution prend quelques minutes (moins de 5).

\section*{Exécuter notre code}
Le code se compile avec le \code{Makefile} déjà présent dans les sources qui nous étaient données. Attention, selon l'installation de \code{camlp5}, et la distribution utilisée, il peut être nécessaire de changer certains chemins vers des modules Caml dans le \code{Makefile}.

Ainsi, il faut lancer \code{make} qui crée un exécutable \code{main}. Celui-ci peut alors être lancé pour faire la résolution des $n$ dames.

\end{document}
