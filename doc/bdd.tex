\documentclass[12pt]{article}
\usepackage[utf8]{inputenc}
\usepackage[T1]{fontenc}
\usepackage[french]{babel}
\usepackage{amsmath,amsfonts,amssymb}
%\usepackage{fullpage}
\usepackage{graphicx}
\usepackage{hyperref}
\usepackage{listings}

\title{Binaey Decision Diagrams}
\author{Aurèle Barrière \& Jérémy Thibault}
\date{6 mai 2016}


\def\question#1{\subsection*{Question #1}}
\def\phix{\varphi\uparrow\up{x}}
\def\ite{\textit{if-then-else}}
\def\P{\mathcal{P}}
\begin{document}
\maketitle
%\newpage

\section*{2\ BDD et forme normale \ite}

\question{1}

On a $\varphi\equiv\phix$ ssi pour toute valuation $V$, $V(\varphi) = V(\phix)$.

Soit $V$ une valuation quelconque. 

\paragraph{Supposons dans un premier temps $V(x) = 1$.} Comme $\phix = (x \wedge\varphi[1/x])\vee(\neg x\wedge\varphi[0/x])$ par définition, on a $V(\phix) = V(\varphi[1/x])$.

De même, comme $V(x) = 1$, on peut remplacer $x$ par 1 dans $\varphi$ sans changer $V(\varphi)$. En effet, par la définition inductive de la valuation d'une formule est un prolongement de la valuation sur les variables. Ainsi $V(\varphi)=V(\varphi[1/x])$ quand $V(x)=1)$.

Ainsi, dans ce cas, on a bien $V(\varphi) = V(\phix)$.

\paragraph{Désormais, supposons $V(x)=0$.} Cette fois-ci, $V(\phix) = V(\varphi[0/x])$.

De même, $V(\varphi) =  V(\varphi[0/x]) = V(\phix)$.

Ainsi, on a bien $\varphi\equiv\phix$.

\question{2}

On commence par montrer que toute formule peut être mise sous forme d'un arbre de décision.

En effet, soit $\varphi$ une formule faisant intervenir $n$ variables. On choisit un ordre sur ces variables que l'on nomme $x_1\dots x_n$.

On construit l'arbre de décision suivant : la racine est $x_1$, et le fils faible et fort de $x_i$ sont $x_{(i+1)}$ pour $i\in\{1\dots n-1\}$. Enfin les fils faibles et forts des noeuds étiquetés $x_n$ sont la valeur de $\varphi$ par la valuation qui à $x_i$ associe 1 si pour aboutir à cette feuille il faut prendre le fils fort à la $i$\up{ème} ligne, et 0 sinon.
%j'espere que c'est clair, c'est un truc tout ballot mais bon faut l'expliquer

On rmarque également que tout sous-arbre d'un arbre de décision reste un arbre de décision. %utile pour l'hérédité

On va ensuite construire la forme \ite\ à partir de cet arbre de décision.

On va montrer que toute formule est équivalente à une formule \ite\ par récurrence sur le nombre de variables impliquées dans la formule.

On pose $\P(n)$ la propriété : ``Toute formule à $n$ variables est équivalente à une formule \ite''.

\paragraph{Initialisation : } 
Pour $n=0$, la formule est nécessairement constante quelle que soit la valuation, égale à 0 ou 1. On peut donc prendre la formule \ite\ 0 ou 1 correspondante.

Pour $n=1$, l'arbre de décision de la formule construit comme précédemment est ainsi réduit à 1 noeud interne (étiqueté par la variable utilisée, appelée $x$), dont les feuilles sont l'évaluation de $\varphi [ 1/x ]$ et $\varphi [ 0/x ]$ : 0 ou 1.

Dans ce cas, la formule est équivalent à $\phix$. C'est bien équivalent par la question précédente, et c'est bien une formule \ite\ car elle est formée à partir de l'opérateur \ite, des constantes 0 et 1 et que la formule test est $x$, une variable non niée.

\paragraph{Hérédité : }
On suppose $\P (n)$ vraie. Montrons $\P (n+1)$.

Soit $\varphi$ une formule utilisant $n+1$ variables $x_1\dots x_{n+1}$. Son arbre d décision construit comme précédemment a pour racine $x_1$, dont les fils faibles et forts sont les racines de sous-arbres de décision de hauteur $n$. Ces deux sous-arbres représentent donc deux formule équivalentex à deux formes \ite\ par hypothèse de récurrence. Ces formules \ite\ seront notées $f$ et $F$,

On peut donc construire la formule \ite\ suivante : $x_1\rightarrow f,F$. C'est bien une formule \ite\ : $f$ et $F$ le sont, et on a utilisé l'opérateur \ite\ avec en formule test $x_1$, une variable non niée.

De plus, $\varphi$ est bien équivalent à cette formule, % à finir

\question{3}


\end{document}
